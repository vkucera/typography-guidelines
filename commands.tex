% User-defined commands

% syntax:
%   \newcommand{cmd name}[# of arguments]{latex expression}

% Languages
%   syntax:
%     \selectlanguage{language}
%     \begin{otherlanguage}{language} long text \end{otherlanguage}
%     \foreignlanguage{language}{short text}
\newcommand{\czech}[1]{{\foreignlanguage{czech}{#1}}} % Czech text
\newcommand{\gr}[1]{{\foreignlanguage{greek}{#1}}} % Greek text

% Fonts
\newcommand{\tn}[1]{\textnormal{#1}} % normal (upright non-bold) text font
\newcommand{\rs}{\fontshape{rs}\selectfont} % serif font
\newcommand{\grs}[1]{\gr{\rs{#1}}} % Greek serif text
\DeclareMathAlphabet{\mathup}{T1}{\familydefault}{m}{n} % upright math font

% Document elements
% definition of a chapter/section not having a number but listed in the table of content
\newcommand{\addchapter}[1]{\clearpage\phantomsection\addcontentsline{toc}{chapter}{#1}} % single-sided printing
\newcommand{\addsection}[1]{\clearpage\phantomsection\addcontentsline{toc}{section}{#1}} % single-sided printing
%\renewcommand{\addchapter}[1]{\cleardoublepage\phantomsection\addcontentsline{toc}{chapter}{#1}} % two-sided printing
%\renewcommand{\addsection}[1]{\cleardoublepage\phantomsection\addcontentsline{toc}{section}{#1}} % two-sided printing

% Text formatting utilities
\newcommand{\qm}[1]{``#1''} % text in English quotation marks
%\let\qm\enquote % localised quotation marks for the current active language (csquotes package)
\newcommand{\hl}[1]{{\color{red}#1}} % red text (xcolor package)
\newcommand{\todo}[1]{\textbf{TODO:} \hl{#1}} % to do reminder
\newcommand{\colorboxmath}[2]{\colorbox{#1}{\ensuremath{#2}}} % coloured math
\newcommand{\urlss}[1]{{\scriptsize \url{#1}}} % URL with small font size
\newcommand{\now}{\today, \currenttime} % current date and time (datetime package)

% Mathematics
\newcommand{\ml}[1]{\(#1\)} % LaTeX inline math mode (math environment)
\newcommand{\me}[1]{\[#1\]} % LaTeX equation math mode (displaymath environment)
\newcommand{\dd}{\mathup{d}} % total differentiation symbol
\newcommand{\df}{\stackrel{\mathup{def}}{=}} % definition symbol
%\newcommand{\slfrac}[2]{\left.#1\right/#2} % solidus after a complex expression
%\renewcommand{\vec}[1]{\mathbf{#1}} % bold vectors
%\renewcommand{\vec}[1]{\vectorsym{#1}} % bold italic vectors (isomath package)
\renewcommand{\vec}[1]{\boldsymbol{#1}} % bold italic vectors
\newcommand{\constpi}{\mathup{\pi}} % constant pi
\renewcommand{\constpi}{\uppi} % constant pi (newtxmath)
\newcommand{\conste}{\mathup{e}} % constant e
\newcommand{\consti}{\mathup{i}} % constant i
\newcommand{\dt}{\mathup{\delta}} % upright lower case delta (symbol for variation or little difference)
\newcommand{\Dt}{\mathup{\Delta}} % upright upper case delta (symbol for great difference)
\renewcommand{\dt}{\updelta} % upright lower case delta (symbol for variation or little difference) (newtxmath)
\renewcommand{\Dt}{\upDelta} % upright upper case delta (symbol for great difference) (newtxmath)
\newcommand{\avg}[1]{\left\langle #1 \right\rangle} % average

% Units
% siunitx package
\sisetup{%
    range-phrase=\text{--}, % use en dash for ranges
    range-units=single, % type unit only once in the ranges
    separate-uncertainty % print uncertainty in format (num ± unc)
    %output-decimal-marker={,}
    %detect-all % typeset numbers with text font instead of mathrm
}%
\DeclareSIUnit\clight{\text{\ensuremath{c}}} % speed of light in vacuum
% \DeclareSIUnit[number-unit-product = ]\percent{\char`\%} % to remove the space before the percentage symbol
% energy
\newcommand{\mev}{\unit{\mega\electronvolt}} % MeV
\newcommand{\gev}{\unit{\giga\electronvolt}} % GeV
\newcommand{\tev}{\unit{\tera\electronvolt}} % TeV
% momentum
\newcommand{\gevc}{\unit[per-mode=symbol]{\giga\electronvolt\per\clight}} % GeV/c
\newcommand{\mevc}{\unit[per-mode=symbol]{\mega\electronvolt\per\clight}} % MeV/c
% mass
\newcommand{\mevcc}{\unit[per-mode=symbol]{\mega\electronvolt\per\clight\squared}} % MeV/c^2
\newcommand{\gevcc}{\unit[per-mode=symbol]{\giga\electronvolt\per\clight\squared}} % GeV/c^2
% other
\newcommand{\fm}{\unit{\femto\metre}} % fm
\newcommand{\fmc}{\unit[per-mode=symbol]{\femto\metre\per\clight}} % fm/c

% Particle symbols
% All particles must be typeset using an upright text font, if possible.
% formatting commands
\newcommand{\particle}[1]{\textup{#1}} % upright text font
\newcommand{\anti}[1]{\ensuremath{\overline{#1}}} % anti-particle
\newcommand{\quark}[1]{\particle{#1}}
\newcommand{\particlegr}[1]{\particle{\gr{#1}}} % particle denoted by a Greek letter
% simple symbols
\newcommand{\proton}{\particle{p}}
\newcommand{\neutron}{\particle{n}}
\newcommand{\electron}{\particle{e}}
\newcommand{\muon}{\particlegr{m}}
\newcommand{\tauon}{\particlegr{t}}
\newcommand{\neutrino}{\particlegr{n}}
\newcommand{\pion}{\particlegr{p}}
\newcommand{\kaon}{\particle{K}}
\newcommand{\lmb}{\particlegr{L}}
\newcommand{\mesonpsi}{\particlegr{y}}
\newcommand{\mesonphi}{\particlegr{f}}
\newcommand{\photon}{\particlegr{g}}
\newcommand{\gluon}{\particle{g}}
\newcommand{\bosonw}{\particle{W}}
\newcommand{\bosonz}{\particle{Z}}
\newcommand{\higgs}{\particle{H}}
% alternative simple symbols using upgreek or newtxmath (lowercase) package
% Comment these out if the text font includes Greek letters.
\renewcommand{\muon}{\ensuremath{\upmu}}
\renewcommand{\tauon}{\ensuremath{\uptau}}
\renewcommand{\neutrino}{\ensuremath{\upnu}}
\renewcommand{\pion}{\ensuremath{\uppi}}
\renewcommand{\photon}{\ensuremath{\upgamma}}
\renewcommand{\lmb}{\ensuremath{\Uplambda}}
\renewcommand{\mesonpsi}{\ensuremath{\uppsi}}
\renewcommand{\mesonphi}{\ensuremath{\upvarphi}}
% alternative simple symbols using newtxmath package
% (Lowercase symbols have same names as in upgreek.)
%   Comment these out if the text font includes Greek letters.
\renewcommand{\lmb}{\ensuremath{\upLambda}}
% derived symbols
\newcommand{\aproton}{\anti{\proton}}
\newcommand{\electronminus}{\ensuremath{\electron^{-}}}
\newcommand{\positron}{\ensuremath{\electron^{+}}}
\newcommand{\muonminus}{\ensuremath{\muon^{-}}}
\newcommand{\muonplus}{\ensuremath{\muon^{+}}}
\newcommand{\tauonminus}{\ensuremath{\tauon^{-}}}
\newcommand{\tauonplus}{\ensuremath{\tauon^{+}}}
\newcommand{\bosonwminus}{\ensuremath{\bosonw^{-}}}
\newcommand{\bosonwplus}{\ensuremath{\bosonw^{+}}}
\newcommand{\pionminus}{\ensuremath{\pion^{-}}}
\newcommand{\pionplus}{\ensuremath{\pion^{+}}}
\newcommand{\pionzero}{\ensuremath{\pion^{0}}}
\newcommand{\kaonminus}{\ensuremath{\kaon^{-}}}
\newcommand{\kaonplus}{\ensuremath{\kaon^{+}}}
\newcommand{\kzeros}{\ensuremath{\kaon^{0}_{\textup{S}}}}
\newcommand{\kzerol}{\ensuremath{\kaon^{0}_{\textup{L}}}}
\newcommand{\dzero}{\ensuremath{\particle{D}^{0}}}
\newcommand{\dplus}{\ensuremath{\particle{D}^{+}}}
\newcommand{\dsubs}{\ensuremath{\particle{D}_{\quark{s}}^{+}}}
\newcommand{\lmbc}{\ensuremath{\lmb_{\quark{c}}^{+}}}
\newcommand{\lmbb}{\ensuremath{\lmb_{\quark{b}}^{0}}}
\newcommand{\bplus}{\ensuremath{\particle{B}^{+}}}
\newcommand{\jpsi}{\mbox{\particle{J}/\mesonpsi}}
\newcommand{\vzero}{\ensuremath{\particle{V}^{0}}}
% nuclei
\newcommand{\pb}{\particle{Pb}} % lead

% Collision systems
\newcommand{\coll}[2]{\mbox{#1--#2}} % en dash notation
\newcommand{\pbpb}{\coll{\pb}{\pb}} % lead--lead (ALICE convention: Pb--Pb)
\newcommand{\ppb}{\coll{\proton}{\pb}} % proton--lead (ALICE convention: p--Pb)
\newcommand{\pbp}{\coll{\pb}{\proton}} % lead--proton (ALICE convention: Pb--p)
\newcommand{\pp}{\coll{\proton}{\proton}} % proton--proton
%\newcommand{\pp}{\mbox{\proton\proton}} % proton--proton (ALICE convention: pp)

% Quantities
\newcommand{\gen}{\tn{gen.}} % generator level label
\newcommand{\rec}{\tn{rec.}} % reconstruction level label
% collisions
\newcommand{\snn}{\sqrt{s_{\particle{NN}}}} % cms energy per nucleon pair
\newcommand{\dnchdeta}{\dd N_{\tn{ch}}/\dd\eta} % dN_ch/deta
\newcommand{\dnchdy}{\dd N_{\tn{ch}}/\dd y} % dN_ch/dy
% kinematics
\newcommand{\pvec}{\vec{p}} % momentum vector
\newcommand{\pt}{p_{\tn{T}}} % p_T, transverse momentum
\newcommand{\pts}[1]{p_{\tn{T,\,}{#1}}} % p_T with additional subscript
